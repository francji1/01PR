% Options for packages loaded elsewhere
\PassOptionsToPackage{unicode}{hyperref}
\PassOptionsToPackage{hyphens}{url}
%
\documentclass[
]{article}
\usepackage{lmodern}
\usepackage{amssymb,amsmath}
\usepackage{ifxetex,ifluatex}
\ifnum 0\ifxetex 1\fi\ifluatex 1\fi=0 % if pdftex
  \usepackage[T1]{fontenc}
  \usepackage[utf8]{inputenc}
  \usepackage{textcomp} % provide euro and other symbols
\else % if luatex or xetex
  \usepackage{unicode-math}
  \defaultfontfeatures{Scale=MatchLowercase}
  \defaultfontfeatures[\rmfamily]{Ligatures=TeX,Scale=1}
\fi
% Use upquote if available, for straight quotes in verbatim environments
\IfFileExists{upquote.sty}{\usepackage{upquote}}{}
\IfFileExists{microtype.sty}{% use microtype if available
  \usepackage[]{microtype}
  \UseMicrotypeSet[protrusion]{basicmath} % disable protrusion for tt fonts
}{}
\makeatletter
\@ifundefined{KOMAClassName}{% if non-KOMA class
  \IfFileExists{parskip.sty}{%
    \usepackage{parskip}
  }{% else
    \setlength{\parindent}{0pt}
    \setlength{\parskip}{6pt plus 2pt minus 1pt}}
}{% if KOMA class
  \KOMAoptions{parskip=half}}
\makeatother
\usepackage{xcolor}
\IfFileExists{xurl.sty}{\usepackage{xurl}}{} % add URL line breaks if available
\IfFileExists{bookmark.sty}{\usepackage{bookmark}}{\usepackage{hyperref}}
\hypersetup{
  pdftitle={Zadani zapoctove ulohy z 01PR},
  pdfauthor={Name1, Name2, NAme3},
  hidelinks,
  pdfcreator={LaTeX via pandoc}}
\urlstyle{same} % disable monospaced font for URLs
\usepackage[margin=1in]{geometry}
\usepackage{graphicx,grffile}
\makeatletter
\def\maxwidth{\ifdim\Gin@nat@width>\linewidth\linewidth\else\Gin@nat@width\fi}
\def\maxheight{\ifdim\Gin@nat@height>\textheight\textheight\else\Gin@nat@height\fi}
\makeatother
% Scale images if necessary, so that they will not overflow the page
% margins by default, and it is still possible to overwrite the defaults
% using explicit options in \includegraphics[width, height, ...]{}
\setkeys{Gin}{width=\maxwidth,height=\maxheight,keepaspectratio}
% Set default figure placement to htbp
\makeatletter
\def\fps@figure{htbp}
\makeatother
\setlength{\emergencystretch}{3em} % prevent overfull lines
\providecommand{\tightlist}{%
  \setlength{\itemsep}{0pt}\setlength{\parskip}{0pt}}
\setcounter{secnumdepth}{-\maxdimen} % remove section numbering

\title{Zadani zapoctove ulohy z 01PR}
\author{Name1, Name2, NAme3}
\date{2022-12-06}

\begin{document}
\maketitle

\hypertarget{zaduxe1nuxed}{%
\subsection{Zadání}\label{zaduxe1nuxed}}

Vytvořte interaktivni Shiny aplikaci, která bude napojena na data z
Eurostatu a bude reagovat na změny parametrů a podle toho vykreslovat
příslušné grafy. Inspirace na
\url{https://ec.europa.eu/eurostat/cache/recovery-dashboard/}

\hypertarget{data}{%
\subsubsection{Data}\label{data}}

Použijte následující datové zdroje a knihovnu \texttt{eurostat}:

\begin{itemize}
\tightlist
\item
  Unemployment by sex and age -- monthly: data code: UNE\_RT\_M
\item
  GDP and main components (output, expenditure and income) - quarterly:
  data code: NAMQ\_10\_GDP
\item
  Business registration and bankruptcy index by NACE Rev.2 activity -
  quarterly: data code: STS\_RB\_Q
\end{itemize}

\hypertarget{popis-uxfakolux16f.}{%
\subsubsection{Popis úkolů.}\label{popis-uxfakolux16f.}}

\begin{itemize}
\item
  Stáhněte data a ta s měsíční frekvencí převeďte na čtvrtletní.
\item
  Vytvorte vpravo ridici panel, kde budete moci zadavat obecne pro
  vsechny 3 grafy:

  \begin{itemize}
  \tightlist
  \item
    Vyber zeme
  \item
    Vyber casoveho obdobi po ctvrtletich
  \item
    Vyber hlavni promenne ktera se zobrazi ve vsech grafech
  \item
    Pro graf cislo 1: Prepinatko mezi prumernou nezamestnanosti a
    mezikvartalni zmenou nezamestnanosti
  \item
    Pro graf cislo 2: Soupatko pro timeshift (posunuti po ctvrtletich v
    nezamestnanosti pro vypocet korelace mezi nezamestnanosti a dalsima
    dvema promennyma) a vyber druhe promenne k prvni hlavni pro
    vykresleni scatterplotu.
  \item
    Pro graf cislo 3: Prepinatko mezi prumerem za vybrane casove obdobi
    a mezi zmenou od zacatku do konce casoveho obdobi.
  \end{itemize}
\end{itemize}

\hypertarget{detailni-popis-jednotlivych-grafu}{%
\subsubsection{Detailni popis jednotlivych
grafu}\label{detailni-popis-jednotlivych-grafu}}

\begin{itemize}
\tightlist
\item
  Graf cislo 1:

  \begin{itemize}
  \tightlist
  \item
    Lomena cara vykreslujici vybrane datove body, kde barva odlisuje
    jednotlive zeme a typ cary odlisuje pohlavi.
  \end{itemize}
\item
  Graf cislo 2:

  \begin{itemize}
  \tightlist
  \item
    Scatterplot, kde barva tecek a prolozene regresni primky odlisuje
    zeme.
  \item
    Pod grafem bude vypis hodnot korelace pro jednotlive zeme.
  \end{itemize}
\item
  Graf cislo 3:

  \begin{itemize}
  \tightlist
  \item
    Mapa s barevnym odlisenim podle honoty zkoumane promenne, ktera bude
    za zoomovana podle toho jake zeme budou vybrany.
  \end{itemize}
\end{itemize}

\end{document}
